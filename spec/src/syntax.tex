\documentclass[../main.tex]{subfiles}
\begin{document}
\section{Syntax}\label{section:syntax}

Details about assignment and arithmetic on floats will follow eventually, but I want to sketch out a theory of combining distributions, first.

Abstractly, you can think of Squiggle as containing two types
\begin{grammar}
  <type> ::= <GenericDist> \alt \texttt{float}
\end{grammar}\label{gram:type}

Where \texttt{float} is ordinary IEEE754 floating point numbers, and $\langle \textit{GenericDist} \rangle$ is as follows

\begin{grammar}
  <GenericDist> ::= <PointSetDist> \alt \texttt{SampleSet} \alt <Symbolic>

  <PointSetDist> ::= <Mixed> \alt <Continuous> \alt <Discrete>

  <Symbolic> ::= \texttt{Normal} \alt \texttt{LogNormal} \alt \texttt{Triangular} \alt \texttt{Beta} \alt \texttt{Uniform} \alt \texttt{Float} \alt \texttt{Exponential} \alt \texttt{Cauchy}
\end{grammar}

From a grammatical perspective \texttt{SampleSet} and all of the alternatives of $\langle \textit{Symbolic} \rangle$ are \textit{black boxes}. They needn't be discussed from this point of view.

Finally, most of the magic happens in $\langle \textit{Expression} \rangle$

\begin{grammar}
  <Factor> ::= <GenericDist> \alt \texttt{float} \alt <Factor> * <Factor> \alt <Factor> .* <Factor>

  <Expression> ::= <Factor> + <Factor> \alt <Factor> .+ <Factor>
\end{grammar}


\end{document}
